\FOT{2}\Seq%
{\def\PageNColumns%
{1}\def\PageColumnSep%
{4\p@}}\Node%
{}\SpS%
{\def\LeftMargin%
{60\p@}\def\TopMargin%
{60\p@}\def\BottomMargin%
{60\p@}\def\PageWidth%
{360\p@}\def\PageNColumns%
{2}\def\PageColumnSep%
{12\p@}}
\SpSOtherBackLeftFooter%
{}
\SpSOtherBackLeftHeader%
{}
\SpSOtherBackCenterFooter%
{}
\SpSOtherBackCenterHeader%
{}
\SpSOtherBackRightFooter%
{}
\SpSOtherBackRightHeader%
{}
\SpSFirstBackLeftFooter%
{}
\SpSFirstBackLeftHeader%
{}
\SpSFirstBackCenterFooter%
{}
\SpSFirstBackCenterHeader%
{}
\SpSFirstBackRightFooter%
{}
\SpSFirstBackRightHeader%
{}
\SpSOtherFrontLeftFooter%
{}
\SpSOtherFrontLeftHeader%
{}
\SpSOtherFrontCenterFooter%
{}
\SpSOtherFrontCenterHeader%
{}
\SpSOtherFrontRightFooter%
{}
\SpSOtherFrontRightHeader%
{}
\SpSFirstFrontLeftFooter%
{}
\SpSFirstFrontLeftHeader%
{}
\SpSFirstFrontCenterFooter%
{}
\SpSFirstFrontCenterHeader%
{}
\SpSFirstFrontRightFooter%
{}
\SpSFirstFrontRightHeader%
{}\DisplayGroup%
{}\insertRule%
{\def\Orientation%
{horizontal}}\Node%
{\def\Element%
{0}}\Node%
{\def\Element%
{1}}\Node%
{\def\Element%
{2}}\Par%
{}The observed total loss per cycle per meter
plotted as a function of the peak amplitude of the magnetic
field. Also shown are the calculated values of the losses assuming
that all the filaments are fully coupled (solid line) and fully
uncoupled (dashed line).\endPar{}\endNode{}\Node%
{\def\Element%
{3}}\Par%
{}It is well known that multifilamentary tapes can be used to reduce
their a.c. losses when their filaments are uncoupled, so as to behave
as many small superconductors rather than one monoblock. However,
given a high enough rate of change of the magnetic field, currents can
be driven across the normal matrix of a multifilament superconductor
resulting in the coupling of the filaments. These currents have an
associated loss which must be added to the losses of the individual
filaments.\endPar{}\endNode{}\Node%
{\def\Element%
{4}}\Par%
{}The outer layer of filaments will become saturated and the next
layer of filaments will begin to conduct. When the entire composite is
saturated, it behaves as a single monoblock, and in this region the
losses can be estimated assuming the composite is a solid
superconductor.\endPar{}\endNode{}\endNode{}\Node%
{\def\Element%
{5}}\Node%
{\def\Element%
{6}}\Par%
{}As the complete field range used in this measurement is well above
this value the tape should be fully coupled.\endPar{}\endNode{}\Node%
{\def\Element%
{7}}\Par%
{}The a.c. losses in a parallel field for the limiting cases of fully
uncoupled and fully coupled filaments can be calculated using Bean's
critical state model for a one dimensional slab.\endPar{}\endNode{}\Node%
{\def\Element%
{8}}\Par%
{}The penetration field is dependent upon the dimensions of the
superconductor and therefore whether the filaments are fully coupled
or uncoupled. For a fully uncoupled tape.\endPar{}\endNode{}\Node%
{\def\Element%
{9}}\Par%
{}The good correspondence between the measured data and the
calculations for fully coupled filaments show that the filaments in
the tape are fully coupled together.\endPar{}\endNode{}\endNode{}\Node%
{\def\Element%
{10}}\Node%
{\def\Element%
{11}}\Par%
{}The measured loss per cycle per unit length of Sample B
as a function of the peak amplitude of the parallel
a.c. magnetic field. Also shown are the losses calculated for fully
coupled filaments (dashed line), fully uncoupled filaments (dotted
line) and fully uncoupled filaments plus the coupling current losses
(solid line).\endPar{}\endNode{}\Node%
{\def\Element%
{12}}\Par%
{}It shows the observed loss per cycle per unit length plotted
(symbols) as a function of the peak magnetic field.\endPar{}\endNode{}\Node%
{\def\Element%
{13}}\Par%
{}It can be seen that these calculated values do not correspond to
the measured values. Calculations were also undertaken assuming that
the filaments had completely decoupled to give 75 individual
conductors with cross-section of 210\Entity{times}20 m.\endPar{}\endNode{}\Node%
{\def\Element%
{14}}\Par%
{}It is seen that these calculations underestimate the actual
measured values. It is known however, that the coupling currents which
flow through the silver matrix also produce losses and therefore may
be responsible for this discrepancy.\endPar{}\endNode{}\endNode{}\endNode{}\endDisplayGroup{}\DisplayGroup%
{\def\Span%
{2}}\insertRule%
{\def\Orientation%
{horizontal}}\Node%
{\def\Element%
{0}}\Node%
{\def\Element%
{1}}\Node%
{\def\Element%
{2}}\Par%
{}The observed total loss per cycle per meter
plotted as a function of the peak amplitude of the magnetic
field. Also shown are the calculated values of the losses assuming
that all the filaments are fully coupled (solid line) and fully
uncoupled (dashed line).\endPar{}\endNode{}\Node%
{\def\Element%
{3}}\Par%
{}It is well known that multifilamentary tapes can be used to reduce
their a.c. losses when their filaments are uncoupled, so as to behave
as many small superconductors rather than one monoblock. However,
given a high enough rate of change of the magnetic field, currents can
be driven across the normal matrix of a multifilament superconductor
resulting in the coupling of the filaments. These currents have an
associated loss which must be added to the losses of the individual
filaments.\endPar{}\endNode{}\Node%
{\def\Element%
{4}}\Par%
{}The outer layer of filaments will become saturated and the next
layer of filaments will begin to conduct. When the entire composite is
saturated, it behaves as a single monoblock, and in this region the
losses can be estimated assuming the composite is a solid
superconductor.\endPar{}\endNode{}\endNode{}\Node%
{\def\Element%
{5}}\Node%
{\def\Element%
{6}}\Par%
{}As the complete field range used in this measurement is well above
this value the tape should be fully coupled.\endPar{}\endNode{}\Node%
{\def\Element%
{7}}\Par%
{}The a.c. losses in a parallel field for the limiting cases of fully
uncoupled and fully coupled filaments can be calculated using Bean's
critical state model for a one dimensional slab.\endPar{}\endNode{}\Node%
{\def\Element%
{8}}\Par%
{}The penetration field is dependent upon the dimensions of the
superconductor and therefore whether the filaments are fully coupled
or uncoupled. For a fully uncoupled tape.\endPar{}\endNode{}\Node%
{\def\Element%
{9}}\Par%
{}The good correspondence between the measured data and the
calculations for fully coupled filaments show that the filaments in
the tape are fully coupled together.\endPar{}\endNode{}\endNode{}\Node%
{\def\Element%
{10}}\Node%
{\def\Element%
{11}}\Par%
{}The measured loss per cycle per unit length of Sample B
as a function of the peak amplitude of the parallel
a.c. magnetic field. Also shown are the losses calculated for fully
coupled filaments (dashed line), fully uncoupled filaments (dotted
line) and fully uncoupled filaments plus the coupling current losses
(solid line).\endPar{}\endNode{}\Node%
{\def\Element%
{12}}\Par%
{}It shows the observed loss per cycle per unit length plotted
(symbols) as a function of the peak magnetic field.\endPar{}\endNode{}\Node%
{\def\Element%
{13}}\Par%
{}It can be seen that these calculated values do not correspond to
the measured values. Calculations were also undertaken assuming that
the filaments had completely decoupled to give 75 individual
conductors with cross-section of 210\Entity{times}20 m.\endPar{}\endNode{}\Node%
{\def\Element%
{14}}\Par%
{}It is seen that these calculations underestimate the actual
measured values. It is known however, that the coupling currents which
flow through the silver matrix also produce losses and therefore may
be responsible for this discrepancy.\endPar{}\endNode{}\endNode{}\endNode{}\endDisplayGroup{}\DisplayGroup%
{}\insertRule%
{\def\Orientation%
{horizontal}}\Node%
{\def\Element%
{0}}\Node%
{\def\Element%
{1}}\Node%
{\def\Element%
{2}}\Par%
{}The observed total loss per cycle per meter
plotted as a function of the peak amplitude of the magnetic
field. Also shown are the calculated values of the losses assuming
that all the filaments are fully coupled (solid line) and fully
uncoupled (dashed line).\endPar{}\endNode{}\Node%
{\def\Element%
{3}}\Par%
{}It is well known that multifilamentary tapes can be used to reduce
their a.c. losses when their filaments are uncoupled, so as to behave
as many small superconductors rather than one monoblock. However,
given a high enough rate of change of the magnetic field, currents can
be driven across the normal matrix of a multifilament superconductor
resulting in the coupling of the filaments. These currents have an
associated loss which must be added to the losses of the individual
filaments.\endPar{}\endNode{}\Node%
{\def\Element%
{4}}\Par%
{}The outer layer of filaments will become saturated and the next
layer of filaments will begin to conduct. When the entire composite is
saturated, it behaves as a single monoblock, and in this region the
losses can be estimated assuming the composite is a solid
superconductor.\endPar{}\endNode{}\endNode{}\Node%
{\def\Element%
{5}}\Node%
{\def\Element%
{6}}\Par%
{}As the complete field range used in this measurement is well above
this value the tape should be fully coupled.\endPar{}\endNode{}\Node%
{\def\Element%
{7}}\Par%
{}The a.c. losses in a parallel field for the limiting cases of fully
uncoupled and fully coupled filaments can be calculated using Bean's
critical state model for a one dimensional slab.\endPar{}\endNode{}\Node%
{\def\Element%
{8}}\Par%
{}The penetration field is dependent upon the dimensions of the
superconductor and therefore whether the filaments are fully coupled
or uncoupled. For a fully uncoupled tape.\endPar{}\endNode{}\Node%
{\def\Element%
{9}}\Par%
{}The good correspondence between the measured data and the
calculations for fully coupled filaments show that the filaments in
the tape are fully coupled together.\endPar{}\endNode{}\endNode{}\Node%
{\def\Element%
{10}}\Node%
{\def\Element%
{11}}\Par%
{}The measured loss per cycle per unit length of Sample B
as a function of the peak amplitude of the parallel
a.c. magnetic field. Also shown are the losses calculated for fully
coupled filaments (dashed line), fully uncoupled filaments (dotted
line) and fully uncoupled filaments plus the coupling current losses
(solid line).\endPar{}\endNode{}\Node%
{\def\Element%
{12}}\Par%
{}It shows the observed loss per cycle per unit length plotted
(symbols) as a function of the peak magnetic field.\endPar{}\endNode{}\Node%
{\def\Element%
{13}}\Par%
{}It can be seen that these calculated values do not correspond to
the measured values. Calculations were also undertaken assuming that
the filaments had completely decoupled to give 75 individual
conductors with cross-section of 210\Entity{times}20 m.\endPar{}\endNode{}\Node%
{\def\Element%
{14}}\Par%
{}It is seen that these calculations underestimate the actual
measured values. It is known however, that the coupling currents which
flow through the silver matrix also produce losses and therefore may
be responsible for this discrepancy.\endPar{}\endNode{}\endNode{}\endNode{}\endDisplayGroup{}\endSpS{}\endNode{}\endSeq{}\endFOT{}